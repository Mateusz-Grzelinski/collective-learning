\section{Opis modelu}

Model przedstawia zdobywanie i rozprzestrzenianie się wiedzy w społeczeństwie. Pierwszym z założeń modelu jest skwantyfikowanie czasu, który jest reprezentowany przez poszczególne tury. Dla odwzorowania procesu założyliśmy, że społeczeństwo zostanie zaprezentowane jako sieć znajomości, przestawiona na grafie nieskierowanym. Graf ów przedstawia możliwe ścieżki przekazywania wiedzy między poszczególnymi osobnikami w populacji (dalej zwanymi agentami) i stanowi element odpowiadający za symulację społeczeństwa zdolnego do wymiany wiedzy. Za pomocą sieci połączeń można zamodelować zarówno strukturę społeczeństwa, jak i różne sposoby dzielenia się wiedzą. Drugim istotnym czynnikiem który braliśmy pod uwagę była niedoskonałość przekazywania wiedzy, którą reprezentuje współczynnik szansy przekazania wiedzy, który określa średnie zdolności poznawczo-kognitywne agentów należących do danego społeczeństwa. Każdy agent posiadać może wiedzę o zasobie, która dla zasymulowania działania jej przekazywania jest charakteryzowania przez jej wiek. Wiek wiedzy determinuje jak daleko może się ona rozprzestrzenić od agenta ją rozpowszechniającego. Ponadto agent posiada przypisanie czyli wiedzę o pojedynczym zasobie który został mu przypisany i do którego ma dostęp na wyłączność. Środowisko jest zamodelowane za pomocą dwu wymiarowej mapy z polami na których może się znajdować pewna ilość zasobu. Agenci nie mający przypisanego zasobu zbierają wiedzę sprawdzając w każdej turze losowe pole, jeżeli znajdzie wiedzę to oznacza to pole jako znalezione (ustawia 0 w ilości zasobu, do swojej wiedzy zapisuje il. Zasobu z pola -1, a sobie przypisuje zasób). Następnie odbywa się wymiana wiedzy. Agenci mający wiedzę przeszukują swoje sąsiedztwo w odległości równej wiekowi ich wiedzy, a poruszając się tylko po węzłach grafu na których agenci już mają przypisanie. Jeżeli w takim sąsiedztwie natrafią na agenta nieposiadającego wiedzy to przekazuję mu ją (nadają przypisanie) i zmniejszają w swej wiedzy ilość zasobów, gdyż przypisany został już jeden z nich innemu agentowi.

Podsumowując Agenci współpracują ze sobą i nie podbierają sobie zasobów, jeżeli agent ma przypisany zasób to nie szuka kolejnych, tylko eksploatuje nadany, a agenci z wiedzą rozprzestrzeniają ją arbitralnie do swych sąsiadów mając na celu powiększenie ilości agentów z wiedzą, a gdy przekażą ją ilości agentów równej ilości zasobów jaką znaleźli, to przestają rozpropagowywać swoją wiedzę i jedynie zbierają zasób i uczestniczę w przeazywaniu wiedzy po sieci powiązań.
