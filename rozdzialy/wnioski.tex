\section{Wnioski}

Zwiększanie parametru p-prawdopodobieñstwa przekazania wiedzy zwiększa szybkość zdobycia wiedzy przez całą populacje niezalełnie od ilości klik jak i ilości agentów w danej klice.

Zwiększanie ilości klik wpływa na zmniejszenie szybkości rozchodzenia się więdzy w populacji. Mniejsza ilość połączeń w grafie, powoduje mniejsze rzeczywiste prawdopodobieństwo przekazania wiedzy efekt warunkowania zdarzeñ. Występują jednak anomalnie takie jak na (zał.1). W specyficznych sytuacjach gdy ilość zasobów dostępnych przewyższa ilość dostępnych aktorów, populacja dąży do eksploatacji jednego źródła, zamiast synchronicznego pobierania z większej ilości źródeł. Problem ten zostaje rozwiązany dzięki większej ilości klik, gdyż mniejsze szansa na rozprzestrzenienie się wiedzy wewnątrz populacji zmusza innych aktorów do poszukiwania zamiast eksploatacji.


Prędkość zdobywania wiedzy jest wprost skorelowana z ilością zasobu dostępnego z konkretnego pola. Jeśli ilość aktorów w klice jest równa iloćci dostępnego zasobu z danej kliki cała klika skupia się na eksploatacji danego źródła przez co zmniejsza ona ogólne prawdopodobieñstwo uzyskania wiedzy przez pozostałą część populacji. Jeśli natomiast ilość dostępnego zasobu z konkretnego źródła przewyższa ilość aktorów w klice, to wiedza zostanie dalej przekazana do pozostałych klik, nie zmuszając ich do ponowego odkrywania danego źródła. W przeciwnej sytuacji, jeśli ilość dostępnego zasobu jest zdecydowanie mniejsza od ilości aktorów w klice pewna część danej kliki, która nie eksploatuje tego źródła, będzie miała mniejszą szansę na zdobycie wiedzy, co wydłuży czas pozyskiwania wiedzy przez całą populację.
