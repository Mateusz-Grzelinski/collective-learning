\section{Opis techniczny}

Do realizacji naszego projektu skorzystaliśmy z języka programowania pyton, w szczególności z biblioteki networkx do tworzenia i obsługi grafów, numpy, do reprezentacji wiedzy, oraz matplotlib.pyplot do tworzenia wykresów danych wynikowych.
Program można uruchomić za pomocą $main.py –h$
Możliwości programu:

\begin{itemize}

\item  max\_iterations - określa maksymalną ilość iteracji którą program wykona, gdzie iteracja jest turą zbierania i dzielenia się wiedzą.
\item  map\_size – określa długość boku kwadratowej mapy zasobów
\item  number\_of\_cells\_with\_resources – oznacza ilość pól na których będą zasoby
\item  value\_of\_resource – oznacza ile zasobów będzie na każdym z wylosowanych pól
\item  number\_of\_cliques – oznacza ilość klik w caveman\_connected\_graph
\item  clique\_size – oznacza wielkość klik (grup agentów które składają się z jednakowej liczby agentów połączonych grafem pełnym)
\item p – parametr oznaczający procentową szansę na przekazanie wiedzy

\end{itemize}

Program składa się z trzech głównych funkcji:

\begin{itemize}

\item Get\_knowledge – odpowiada za znajdowanie wiedzy przez agentów,
\item Share\_knowledge – odpowiada za przekazywanie sobie wiedzy przez agentów.
\item Iterate\_knowledge – odpowiada za czas, przebieg poszczególnych etapów rozwoju wiedzy w populacji, a także za zapisywanie wyników.

\end{itemize}
