\documentclass[11pt]{aghdpl}
% \documentclass[en,11pt]{aghdpl}  % praca w języku angielskim

% Lista wszystkich języków stanowiących języki pozycji bibliograficznych użytych w pracy.
% (Zgodnie z zasadami tworzenia bibliografii każda pozycja powinna zostać utworzona zgodnie z zasadami języka, w którym dana publikacja została napisana.)
\usepackage[english,polish]{babel}

% Użyj polskiego łamania wyrazów (zamiast domyślnego angielskiego).
\usepackage{polski}

\usepackage[utf8]{inputenc}

% dodatkowe pakiety

\usepackage{mathtools}
\usepackage{amsfonts}
\usepackage{amsmath}
\usepackage{amsthm}

% --- < bibliografia > ---
\usepackage[
style=numeric,
sorting=none,
%
% Zastosuj styl wpisu bibliograficznego właściwy językowi publikacji.
language=autobib,
autolang=other,
% Zapisuj datę dostępu do strony WWW w formacie RRRR-MM-DD.
% urldate=iso,
% Nie dodawaj numerów stron, na których występuje cytowanie.
backref=false,
% Podawaj ISBN.
isbn=true,
% Nie podawaj URL-i, o ile nie jest to konieczne.
url=false,
%
% Ustawienia związane z polskimi normami dla bibliografii.
maxbibnames=3,
% Jeżeli używamy BibTeXa:
backend=bibtex
]{biblatex}

\usepackage{csquotes}
% Ponieważ `csquotes` nie posiada polskiego stylu, można skorzystać z mocno zbliżonego stylu chorwackiego.
\DeclareQuoteAlias{croatian}{polish}

\addbibresource{bibliografia.bib}

% Nie wyświetlaj wybranych pól.
%\AtEveryBibitem{\clearfield{note}}


% ------------------------
% --- < listingi > ---

% Użyj czcionki kroju Courier.
\usepackage{courier}

\usepackage{listings}
\lstloadlanguages{TeX}

\lstset{
	literate={ą}{{\k{a}}}1
           {ć}{{\'c}}1
           {ę}{{\k{e}}}1
           {ó}{{\'o}}1
           {ń}{{\'n}}1
           {ł}{{\l{}}}1
           {ś}{{\'s}}1
           {ź}{{\'z}}1
           {ż}{{\.z}}1
           {Ą}{{\k{A}}}1
           {Ć}{{\'C}}1
           {Ę}{{\k{E}}}1
           {Ó}{{\'O}}1
           {Ń}{{\'N}}1
           {Ł}{{\L{}}}1
           {Ś}{{\'S}}1
           {Ź}{{\'Z}}1
           {Ż}{{\.Z}}1,
	basicstyle=\footnotesize\ttfamily,
}

% ------------------------

\AtBeginDocument{
	\renewcommand{\tablename}{Tabela}
	\renewcommand{\figurename}{Rys.}
}

% ------------------------
% --- < tabele > ---

\usepackage{array}
\usepackage{tabularx}
\usepackage{multirow}
\usepackage{booktabs}
\usepackage{makecell}
\usepackage[flushleft]{threeparttable}

% defines the X column to use m (\parbox[c]) instead of p (`parbox[t]`)
\newcolumntype{C}[1]{>{\hsize=#1\hsize\centering\arraybackslash}X}


%---------------------------------------------------------------------------

\author{Mateusz Grzeliński, Kornel Wilk, Mateusz Szymkowski}
\shortauthor{M. Grzeliński, K. Wilk, M. Szymkowski}

\titlePL{Uczenie populacji w oparciu o różne metody przekazywania wiedzy}
\titleEN{Collective learning}
\shorttitlePL{}
\thesistype{Modelowanie i symulacja systemów}
\degreeprogramme{Informatyka}
\date{2018}
\department{Katedra Informatyki Stosowanej}
\faculty{Wydział Elektrotechniki, Automatyki,\protect\\[-1mm] Informatyki i Inżynierii Biomedycznej}
\acknowledgements{}


\setlength{\cftsecnumwidth}{10mm}

%---------------------------------------------------------------------------
\setcounter{secnumdepth}{4}
\brokenpenalty=10000\relax

\begin{document}

\titlepages

% Ponowne zdefiniowanie stylu `plain`, aby usunąć numer strony z pierwszej strony spisu treści i poszczególnych rozdziałów.
\fancypagestyle{plain}
{
        % Usuń nagłówek i stopkę
        \fancyhf{}
        % Usuń linie.
        \renewcommand{\headrulewidth}{0pt}
        \renewcommand{\footrulewidth}{0pt}
}

\setcounter{tocdepth}{2}
\tableofcontents
\clearpage

\chapter{Wstęp}
\section{Opis problemu}

\section{Terminologia}

\section{artykuły naukowe}
\begin{enumerate}
\item Impact of social media on collective learning
Collective learning jest zarówno konwersacją bezpośrednią jak i na odległość.
Wcześniejsza w rozwoju wersja to collectiva learning na spotkaniach służących wymianie wiedzy. Chęć skorzystania z danego sposobu nauki zależy od tego jak wygodnie się z niego korzysta i na ile uważa się, że poszerza on wiedzę uczącego się.
Social media pozytywnie wpływają na chęć nauki i rezultaty uzyskiwane w jej trakcie.
\item Conceptualizing Collective Learning and Knowledge Building in the context of migration and development
Typy wiedzy:
1.	Wiedza zdobyta przez doświadczenie, rozumienie danej rzeczy
2.	Świadomość istnienia czegoś
3.	Wiedza jako wyprowadzona przez rozumowanie
4.	Posiadanie informacji lub zostanie nauczonym
Można podzielić też ze względu na sposób przechowywania:
1.	Wewnętrzna- zdobyta przez doświadczenie lub wymianę wiedzy, polega na między innymi na rozumieniu danego zjawiska, nie da się jej ująć językiem czy formułami matematycznymi
2.	Zewnętrzna- zapisana w jakiś sposób na zewnętrznym nośniku, nie jest przechowywana w umyśle, lecz np.: w książce, malowidłach itp. Jest ona bardziej uniwersalna, gdyż jej celem jest bycie zrozumiałą dla szerokiego kręgu odbiorców, wyrażona językiem, bądź regułami
Mądrość jest znajomością powodów, reguł, schematów które można wyprowadzić z danych, ale które również mogą służyć do przewidywania rzeczywistości.
Nauka może być przeprowadzona przez:
1.	Rozumowanie i dzielenie wiedzy
2.	Działanie

Collectiva learning ma na celu podzielenie się wiedzą przez uczestników, wymianę wiedzy, danych, refleksja nad nimi i efekt/odpowiedź

\end{enumerate}

% przykad:
\include{rozdzialy/ogolny-opis-systemu}

% \appendix
% \include{dodatekA}
% \include{dodatekB}
% itd.

\printbibliography

\end{document}
